\documentclass[runningheads]{llncs}

\usepackage[utf8]{inputenc}
\usepackage{graphicx}
\usepackage{todonotes}
\usepackage{multirow}
\usepackage{array,booktabs,tabularx,ragged2e,caption}
\newcolumntype{M}[1]{>{\centering\arraybackslash}m{#1}}
\newcolumntype{C}[1]{>{\Centering}m{#1}}
\renewcommand\tabularxcolumn[1]{C{#1}}

\newcommand{\rj}[1]{\textcolor{teal}{#1}}
\newcommand{\myblue}[1]{{\color{blue}{#1}}}
\newcommand{\myred}[1]{{\color{red}{#1}}}
\newcommand{\mygreen}[1]{{\color{ForestGreen}{#1}}}

\usepackage{multicol,lipsum}
\renewcommand{\thefigure}{9.\arabic{figure}}
\renewcommand{\thetable}{9.\arabic{table}}

\title{Finding datasets in publications: The University of Paderborn approach}
% \title{DICE @ Rich Context Competition 2018 -- Combining Embeddings and Conditional Random Fields for Research Dataset, Field and Method Recognition and Linking}
\titlerunning{DICE @ Rich Context Competition 2018}
\author{Rricha Jalota\orcidID{0000-0003-1517-6394} \and Nikit Srivastava \and Daniel Vollmers \and René Speck \and Michael R\"oder \and Ricardo Usbeck\orcidID{0000-0002-0191-7211} \and Axel-Cyrille {Ngonga Ngomo}\orcidID{0000-0001-7112-3516}}
\authorrunning{Jalota et al.}


\institute{
DICE Group, CS Department, Paderborn University, Germany\\
\email{firstname.lastname@uni-paderborn.de}
}

\begin{document}
%\lipsum[1]
\maketitle

\begin{abstract}
    The steadily increasing number of publications available to researchers makes it difficult to keep track of the state of the art. In particular, tracking the datasets used, topics addressed, experiments performed and results achieved by peers becomes increasingly tedious. Current academic search engines %such as Google Scholar, %\footnote{https://scholar.google.de}, Semantic Scholar, etc. 
    render a limited number of entries pertaining to this information. However, having this knowledge would be beneficial for researchers to become acquainted with all results and baselines relevant to the problems they aim to address. With our participation in the NYU Coleridge Initiative’s Rich Context Competition, we aimed to provide approaches to automate the discovery of datasets, research fields and methods used in publications in the domain of Social Sciences. We trained an Entity Extraction model based on Conditional Random Fields and combined it with the results from a Simple Dataset Mention Search to detect datasets in an article.  For the identification of Fields and Methods, we used word embeddings. In this chapter, we describe how our approaches performed, their limitations, some of the encountered challenges and our future agenda. 
\end{abstract}

% \section{Introduction}
% \subsection{Rich Context Competition}
% The goal of the Rich Context Competition\footnote{\url{https://coleridgeinitiative.org/richcontextcompetition}}, organized by the New York University under their Coleridge Initiative, was to automate the discovery of research datasets, associated research methods and fields in research publications belonging to the domain of Social Sciences. It was carried out in two phases. 
% In the first phase (Phase-1), we were provided with a list of datasets along with their metadata (dataset vocabulary), a training corpus of 5000 publications containing publication metadata (2500 of them were labeled) and an additional dev fold of 100 publications. Apart from this, we were also given Social Science Methods and Fields vocabularies by SAGE Publications\footnote{\url{https://uk.sagepub.com/en-gb/eur/home}}. To carry out Phase-1 evaluation, a separate corpus of 5000 labeled publications was held back.
% %\todo[inline]{write about output files}

% In the second phase (Phase-2), in addition to the Phase-1 data for training, we were provided with the Phase-1 holdout set consisting of 5000 labeled publications and an additional corpus of 5000 unlabeled publications. The evaluation of the second phase was carried out by the organizers on another corpus that contained 5000 unlabeled publications. Note that the labeled data in both the phases was for Dataset Detection only. There was no ground truth for Research Methods and Fields. 
% %Note that, the labeled data in both the phases was for Dataset Detection only. There was no ground truth for Research Methods and Fields. 


\setcounter{figure}{0}
\setcounter{table}{0}

\section{Literature Review}
Previous works on information retrieval from scientific articles are mainly seen in the field of Bio-medical Sciences and Computer Science, with systems~\cite{DBLP:journals/ploscb/WestergaardSTJB18} built using the MEDLINE\footnote{\url{https://www.nlm.nih.gov/bsd/medline.html}} abstracts, full-text articles from PubMed Central\footnote{\url{https://www.ncbi.nlm.nih.gov/pmc/}} or ACL Anthology
dataset\footnote{\url{https://www.aclweb.org/anthology/}}. The documents belonging to the above-mentioned datasets follow a similar format, and thus, several metadata and bibliographical extraction frameworks like CERMINE~\cite{tkaczyk2014cermine} have been built on them. However, since articles belonging to the domain of Social Sciences do not follow a standard format, extracting key sections and metadata using already existing frameworks like GROBID~\cite{lopez2009grobid}, ScienceParse\footnote{\url{https://github.com/allenai/science-parse}} or ParsCit~\cite{councill2008parscit} did not seem as viable options, majorly because these systems were still under development and lacked certain desired features. Hence, building upon the approach of Westergaard et. al~\cite{DBLP:journals/ploscb/WestergaardSTJB18}, we built our own sections-extraction framework for dataset detection and research fields and methods identification. 

Apart from content and metadata extraction, key-phrase or topic extraction from scientific articles has been another emerging research problem in the domain of information retrieval from scientific articles. Jansen et al.~\cite{jansen2016extracting} extracted core claims from scientific articles by first detecting keywords and key-phrases using rule-based, statistical, machine learning and domain-specific approaches and then applying document summarization techniques. For characterizing a research work in terms of its focus, application domain and techniques used, Gupta et al.~\cite{gupta2011analyzing} proposed applying semantic extraction patterns to the dependency trees of sentences in an article's abstract. On the other hand, to thematically represent scientific articles and for ranking the extracted key-phrases,  Mahata et al.~\cite{mahata2018key2vec} devised an approach for processing text documents to train phrase embeddings. 

The problem of dataset detection and methods and fields identification is not only different from the ones mentioned above, but also our approach for tackling it is radically disparate. The following sections describe our approach in detail.

\section{Project Architecture}

\begin{figure}[!htb]
    \centering
    \includegraphics[width=\textwidth]{images/flowchart_paper.pdf}
   %\includegraphics[scale=0.46]{images/flowchart_paper.pdf}
    \caption{Data Flow Pipeline: Red lines depict the flow of given and generated files between components whereas black lines represent the generation of final output files}
    % \scriptsize{}
    \label{fig:flowchart}
\end{figure}

Our pipeline (shown in Figure~\ref{fig:flowchart}) consisted of three main components: 1) Preprocessing, 2) Fields and Methods Identification and 3) Dataset Extraction. The Preprocessing module read the text from publications and generated some additional files (see Section~\ref{preprocess} for details). These files along with the given Fields and Methods vocabularies were used to infer Research Fields and Methods from the publications. Then, the information regarding Research Fields was passed onto the Dataset Detection module and using the Dataset Vocabulary, Dataset Citations and Mentions were identified. The following sections provide a detailed overview of each of these components. 

\section{Preprocessing} \label{preprocess}

As discussed in Chapter 5, the publications were provided in two formats: PDF and text. %the form of both PDF and text. 
For Phase-1, we used the given text files, however during Phase-2, we came across many articles in the training files that had not been properly converted to text and contained mostly non-ASCII characters. To work with such articles, we relied on the open source tool \texttt{pdf2text} from \texttt{poppler suite}\footnote{\label{poppler}\url{https://manpages.debian.org/testing/poppler-utils}} to extract text from PDFs. The \texttt{pdf2text} command served as the first preprocessing step and was called as a subprocess from within a python script. It was used with \texttt{-nopgbrk} argument to generate the text files. 

\smallskip

Once we had the text files, we followed the rule-based approach as proposed by Westergaard et al.~\cite{DBLP:journals/ploscb/WestergaardSTJB18} for pre-processing. The following series of operations based mostly on regular expressions were performed:
\begin{itemize}
    \item Words split by hyphens were de-hyphenated %\todo[inline]{should you be more specific than 'handled'? Unclear what that means};
    \item Irrelevant data was removed (i.e., equations, tables, acknowledgment, references);
    \item \raggedright Main sections (i.e., abstract, keywords,
    JEL-Classification, methodology/data, summary, conclusion) were identified and extracted;
    \item Noun phrases from these sections were extracted (using the python library, spaCy\footnote{\url{https://github.com/explosion/spaCy}}).
\end{itemize}

%\smallskip
We came up with the heuristics for identifying the main sections after going through the articles from different domains in the training data. We collected the surface forms for the headings of all major sections (abstract, keywords, introduction, data, approach, summary, discussion) and applied regular expressions to search for them and separate them from one another. The headings and their corresponding content were stored as key-value pairs in a file. %, which were used later for identifying noun-phrases in relevant sections. 
For generating noun-phrases, this file was parsed and for all the values (content) in key-value (heading-content) pairs, a spaCy object, \texttt{doc}, was created sentence-wise. Using the built-in function for extracting noun chunks ({\texttt{doc.noun\_chunks}}), we generated key-value pairs of heading and noun-phrases found in the content and stored them in another file. This file was later used for fields and methods identification.

To determine how well our approach performed in distinguishing sections, we evaluated it on the articles in the validation dataset. During evaluation, we figured out the limiting cases of our approach. A section could not be differentiated either when there was no explicit mention of any of its surface forms or if there were multiple mentions of the surface forms in the articles. For instance, in the validation dataset (see Table~\ref{tab:sections}), keywords were not extracted from 13 articles because of no explicit mention of the term 'keywords' or its variants. On manual inspection, we found keywords were actually not mentioned in these 13 articles. In the remaining articles where the keywords were present, our algorithm could not detect them from 1 article. 
% Table~\ref{tab:sections} shows the number of identified sections in validation data. 
For brevity, we have reported only four main sections in Table~\ref{tab:sections}: title, abstract, keywords and methodology/data, since these are the ones getting preferential treatment in methods and fields identification. If a section was not found in the article (because of no explicit mention of any of the surface forms), then only the sections that could be detected were extracted. The remaining content was saved as \texttt{reduced\_content} after cleaning and noun-phrases were extracted from it to prevent loss of any meaningful data. 

\begin{table}[!htb]
    \captionsetup{justification=centering,margin=1.2cm}
    \caption{Evaluation of identification of sections in Validation Data (100 articles)} \label{tab:sections}
    \begin{tabular}{C{4cm} C{3.5cm} C{4cm}} \hline
        \textbf{Sections} & \textbf{No explicit mention} & \textbf{Mentioned but not distinguished}  \\ \hline
        Title & 0 & 0 \\ \hline
        Keywords & 13 & 1\\ \hline
        Abstract & 0 & 1\\ \hline
        Methodology/Data & 18 & 4 \\ \hline
    \end{tabular}
\end{table}


In addition to the main sections, we also extracted PDF metadata using \texttt{pdfinfo} service from the \texttt{poppler suite} library. The metadata very often contained the keywords and subject of an article, which was helpful in those cases where the keywords were not found by the regular expression. \\
%To extract PDF metadata that seldom contained keywords and subject of an article, the open source tool, \texttt{pdfinfo} from \texttt{poppler suite}\footnote{https://manpages.debian.org/testing/poppler-utils/pdfinfo.1.en.html}, was employed.   
In the end, the preprocessing module generated four text files for a publication: PDF-converted text, PDF-metadata, processed articles containing relevant data, and noun phrases from the relevant sections, respectively. These files were then passed on to the other two components of the pipeline, which have been discussed below. %for training and evaluating the models.

\section{Approach}

%In this section, we will describe in detail our approach to Research Fields and Methods Identification, and Dataset Extraction.

\subsection{Research Fields and Methods Identification}
%\subsubsection{Preprocessing} 
\subsubsection{Vocabulary Generation and Model Preperation}
\smallskip
\begin{enumerate}
    \item \textbf{Research Methods Vocabulary}: In Phase-1 of the challenge, we used the given methods vocabulary. However, the feedback that we received from Phase-1 evaluation gave more emphasis to statistical methods used by the authors, references to the time scope, unit of observation, and regression equations rather than the means used to compile the data, i.e., surveys. Since the given methods vocabulary was not a complete representation of statistical methods and also consisted terms depicting surveys, in Phase-2, we decided to create our own Research Methods Vocabulary using Wikipedia and DBpedia.\footnote{\url{https://wiki.dbpedia.org/services-resources/ontology}} We manually curated a list of all the relevant statistical methods from Wikipedia\footnote{\url{https://en.wikipedia.org/wiki/Category:Statistical\_methods}} and fetched their descriptions from the corresponding DBpedia resources. 
    For each label in the vocabulary, we extracted noun phrases from its description and added them to the vocabulary. 
    Please refer Table~\ref{tab:vocab} for examples.
    \begin{table}
    \caption{Examples from manually-curated methods vocabulary} \label{tab:vocab}
    \begin{tabular}{C{1.5cm} C{5cm} C{5cm}} \hline
        \textbf{Label} & \textbf{Description} & \textbf{Noun Phrases from Description}  \\ \hline
        Political forecasting & Political forecasting aims at predicting the outcome of elections. & Political forecasting, the outcome, elections\\ \hline
         Nested sampling algorithm & The nested sampling algorithm is a computational approach to the problem of comparing models in Bayesian statistics, developed in 2004 by physicist John Skilling.
 & algorithm, a computational approach, the problem, comparing models, Bayesian statistics, physicist John Skilling \\ \hline
    \end{tabular}
    \end{table}
    \smallskip
    \item \textbf{Research Fields Vocabulary}: For both the phases, we used the given research fields vocabulary and, just like the methods vocabulary, supplemented it with the  noun phrases from the description of the research field labels. However, since our phase-1 model seemed to confuse fields with methods, for Phase-2, we additionally created a stopword-list of terms that didn't contain any domain-specific information, such as; Mixed Methods, Meta Analysis, Narrative Analysis and the like.  
    \smallskip
    \item \textbf{Word2Vec Model generation}: In this pre-processing step, we used the above-mentioned vocabulary files containing noun phrases to generate a vector model for both research fields and methods. %for each research field and method in the vocabularies.
    The vector model was generated by using the labels and noun phrases from the description of the available research fields and methods to form a sum vector. 
    % description of the available research fields and methods and then using the noun phrases present in them to form a sum vector.
    The sum vector was basically the sum of all the vectors of the words present in a particular noun phrase. %“GoogleNews-vectors-negative300.bin” 
    \emergencystretch 3em {The pre-trained Word2Vec model \texttt{GoogleNews-vectors-negative300.bin}~\cite{DBLP:journals/corr/abs-1301-3781} was used to extract the vectors of the individual words.}
    \smallskip
    \item \textbf{Research Method training results creation}: For research methods, we generated an intermediate %a research methods
    result file with the publications present in the training data. %The results were 
    It was generated using a \texttt{naïve finder algorithm} which, for each publication, selected the research method with the highest cosine similarity to any of its noun phrase’s vectors. This file was later used to assign weights to research methods using Inverse Document Frequency.
\end{enumerate}
% \textbf{Word2Vec Model generation} \\
% In this pre-processing step, we used the given vocabulary files for research fields and methods to generate a vector model for each research field and method in the vocabulary. The vector model was generated by using the labels and description of the available research fields and methods and then using the noun phrases present in them to form a sum vector. The sum vector was basically the sum of all the vectors of the words present in a particular noun phrase. “GoogleNews-vectors-negative300.bin” Word2vec model was used to extract the vectors of the individual words.

% \bigskip

% \textbf{Research Method training results creation} \\
% In this step, we generated a research methods result file for the publications present in the training set. The results were generated using a naïve “finder” algorithm which for each publication, selects the research method that has the highest cosine similarity to any of its noun phrase’s vector. This result is later used to assign weights to Research Methods using Inverse Document Frequency.

\subsubsection{Processing with Trained Models} 
%\textbf{Processing} \\
\smallskip
\begin{itemize}
    \item \textbf{Finding Research Fields and Methods:} To find the research fields and methods for a given list of publications, we performed the following steps: (At first, Step 1 was executed for all the publications, thereafter Step 2 and 3 were executed iteratively for each publication).
%\todo[inline]{enumerate itemize}
\begin{enumerate}
    \item \textbf{Naïve Research Method Finder run} - In this step, we executed the \texttt{naïve research method finding algorithm} (i.e. selected a research method based on the highest cosine similarity between vectors) against all the current publications and then merged the results with the existing result from the \texttt{research methods' preprocessing step}. The combined result was then used to generate IDF weight values for each \texttt{research method}, to compute the significance of recurring terms.
    \smallskip
    % \smallskip
    \item \textbf{IDF-based Research Method Selection} - %Similar to the step above, 
    We re-ran the algorithm to find the closest research method to each noun phrase and then sorted the pairs based on their weighted cosine similarity. The weights were taken from the IDF values generated in the first step and the manual weights assigned (section-wise weighting). %The intuition behind IDF was to normalise the biasing of the vector models for frequently occuring terms. 
    Here, the noun phrases that came from the methodology section and from the methods listed in JEL-classification (if present) were given a higher preference. The pair with the highest weighted cosine similarity was then chosen as the Research Method of the article.
    \item \textbf{Research Field Finder run} - In this step, we first found the closest research field from each noun phrase in the publication. Then we selected the Top N (= 10) pairs that had the highest weighted cosine similarity. Afterwards, the noun phrases that had a similarity score less than a given threshold (= 0.9) were filtered out. The end-result was then passed on to a post-processing algorithm. \\ %Note that, 
    For weighted cosine similarity, the weights were assigned manually based on the section of publication from which the noun phrases came. In general, noun phrases from title and keywords (if present) were given a higher preference than other sections, since usually these two sections hold the crux of an article. Note, if sections could not be discerned from an article, then noun phrases from the section, reduced\_content (see section \ref{preprocess}), were used to find both fields and methods.
    \smallskip
    \item \textbf{Research Field Selection} - The top-ranked term from the %previous step's 
    result of step 3, which was not present in the stopword-list of irrelevant terms, was marked as the research field of the article.
    \smallskip
\end{enumerate}
\end{itemize}

% Experimental results checklist - https://arxiv.org/pdf/1909.03004.pdf
The experimental set-up and average training times (ATT) have been reported in Table \ref{tab:setup}:

\begin{table}[!htb]
    \centering
    \captionsetup{justification=centering,margin=1.2cm}
    \caption{Experimental set-up for training the word2vec models for Research Field (RF) and Methods (RM) Identification}
    \label{tab:setup}
    \begin{tabular}{|C{5cm} | C{7cm} |} \hline
        Computing Infrastructure & macOS, 2 GHz Intel Core i7 processor, 4 cores \smallskip RAM 16 GB 1600 MHz  \\ \hline
        ATT - RF model & 3m 21s \\ \hline
        ATT - RM model & 3m 19s\\ \hline
        Link to Implemented Code & \url{https://github.com/nikit91/Jword2vec/tree/rich-context}\\ \hline
    \end{tabular}
\end{table}


% \subsubsection{Running the model against publications}
% The model is run as a jar file from inside a shell script which is executed in the python file.


% Arguments accepted by the jar file - 
% The path to the noun phrases generated from articles 
% The path to the vocabularies  (two separate arguments for each)
% The path to research methods results on training dataset for IDF calculation
% The path where the output files must be stored (two separate arguments for each)


% It generates two files - research$_$fields$_$results.json and research$_$methods$_$results.json, which are used to create the final output files - methods.json and research$_$fields.json. 


\subsection{Dataset Extraction}
For identifying the datasets in a publication, we followed two approaches and later combined results from both. Both the approaches have been described below.

\begin{enumerate}
    \item \textbf{Simple Dataset Mention Search:}
We chose the dataset citations from the given Dataset Vocabulary that occurred for one dataset only and used these unique mentions to search for the corresponding datasets using regular expressions in the text documents. Then, we computed a frequency distribution of the datasets. As can be seen from Figure~\ref{fig:graph}, certain dataset citations occurred more often than others. This is because while searching for dataset citations, apart from the dataset title, the corresponding mention\_list from Dataset Vocabulary was also considered, which contained many commonly occurring terms like `time', `series', `time series', `population' etc. Therefore, we filtered out those dataset citations that occurred more than a certain threshold value (=1.20) multiplied by the median of the frequency distribution and that had less than 3 distinct mentions in a publication. The remaining citations were written to an interim result file. Table~\ref{tab:simple} depicts the improvement in performance of Simple Dataset Mention Search with the inclusion of filtering. The filtering process improved the F1-measure by 42.86\%. Note, as the validation data consisted of only 100 articles, changing the threshold value to 1.10 or 1.30 didn't result in any significant change, hence we have maintained a constant threshold value of 1.20 in our comparison table. 
\begin{table}[!htb]
    \captionsetup{justification=centering,margin=1.2cm}
    \caption{Performance of Simple Dataset Mention Search against Validation Data.} \label{tab:simple}
    \begin{tabular}{C{1.5cm} C{3.5cm} C{3.5cm} C{3.5cm}} \hline
        \textbf{Metrics} & \textbf{without filtering} & \textbf{ Threshold=1.20, mentions $<$ 3} & \textbf{ Threshold=1.20, mentions $<$ 4}  \\ \hline
        Precision & 0.09 & 0.71 & 0.09\\ \hline
        Recall & 0.28 & 0.12  & 0.28\\ \hline
        F1-score & 0.14 & \textbf{0.20} & 0.14\\ \hline 
    \end{tabular}
\end{table}
    
\begin{figure}[!htb]
    \centering
    \includegraphics[scale=0.45]{images/freq.pdf}
    \caption{Frequency Distribution of Dataset Citations}
    \label{fig:graph}
\end{figure}
%\smallskip

\item \textbf{Rasa-based Dataset Detection:}
In our second approach, we trained an entity extraction model based on conditional random fields (CRF) using Rasa NLU~\cite{DBLP:journals/corr/abs-1712-05181}. For training the model we used the Spacy Tokenizer\footnote{https://spacy.io/api/tokenizer} for the preprocessing step. For Entity Recognition we used BILOU tagging and used 50 iterations to train the CRF. We used the Part of Speech tags, the case of the input tokens and the suffixes of the tokens as input features for the CRF model. 
We particularly tested two configurations for training the CRF-based Named Entity Recognition (NER) model. In Phase-1, the 2500 labeled publications from the training dataset were used for training the Rasa NLU\footnote{\url{https://rasa.com/docs/nlu}} model. Later in Phase-2, when the Phase-1 holdout corpus was released, we combined its 5000 labeled publications with the previously given 2500 labeled publications and then retrained the model again with these 7500 labeled publications. %The 7500 labeled publications from phase 1 corpus  (after preprocessing) are used for generating the training data for Rasa NLU. Specifications for model training are given below. 
\\
\textbf{Running the CRF-Model:} The trained model was run against the preprocessed data to detect dataset citations and mentions. Only the entities that had a confidence score greater than a certain threshold value (= 0.72) were considered as dataset mentions. A dataset mention was considered as a citation only if it was found in the given Dataset Vocabulary (via string matching either with a dataset title or any of the terms in a dataset `mention\_list') and if it belonged to the research field of the article. %were considered as datasets and written to files. 
To check if a dataset belonged to the field of research, we found the cosine similarity of the terms in the ‘subjects’ field of the Dataset Vocabulary with the keywords and the identified Research Field of the article. 
\smallskip
\item \textbf{Combining the two approaches:}
The output generated by the Rasa-based approach was first checked for irrelevant citations before a union was performed to combine the results. 
This was done by checking if a given dataset\_id occured more than a threshold value (= 1.20) multiplied by the median of the frequency distribution (same as the filtering process of the Simple Dataset Mention Search). %After filtering out such citations, the results of both the approaches were combined by taking a union  % two approaches above was checked for fake mentions before being written to the final output file data$_$set$_$citations.json, and data$_$set$_$mentions.json in data/output.  This is done by calculating the frequency of dataset mentions and removing mentions that occur more than a threshold$_$value * median of dataset$_$frequency.
\end{enumerate} 

Note that, the threshold values mentioned above were set after some experiments of trial and testing. For dataset extraction, the goal was to keep the number of false positives low while not compromising the true positives. %\todo[inline]{Not sure if you mean 'compromising with' or 'compromising'. The two mean different things. The first means altering with (using) the true positives and the second means altering the true positives themselves. ??} 
For research methods and fields, a manual evaluation (see the next section for details) was done to test if the results made sense with the articles.

\section{Evaluation}
We performed a quantitative evaluation for Dataset Extraction using the evaluation script provided by the competition organizers (refer Chapter 5 for more details). This evaluation (see Table \ref{tab:dataset}) was carried out against the validation data, wherein we compared four different configurations. As can be inferred from the table, %\ref{tab:dataset}, 
there was only a slight increase in performance for the Rasa-based model, when the training samples were increased. However, combining it with the Simple Dataset Mention Search, increased the performance by \emph{19.42\%}. Interestingly, there was no improvement in performance in the combined approach even when the training samples for the Rasa-based model were increased. This might be because of the removal of frequently-occuring terms from the Rasa-generated output, based on the frequency distribution of dataset mentions as computed in the Simple Dataset Mention Search.  \\
%\todo[inline]{explain why no increase - no uniformity in labeled data}

%\todo[inline]{multirow heading for phase-1 and phase2 in table 1}

%\begin{center}
    %\myred{Quantitative Evaluation of Datasets against Validation data} \\
    %\bigskip
\begin{table}
    \captionsetup{justification=centering,margin=1.2cm}
    \caption{Quantitative Evaluation of Datasets against Validation Data. (The numbers inside brackets indicate training samples)} \label{tab:dataset}
    \begin{tabular}{ M{2.2cm} | M{2.3cm} |  M{2.2cm} M{2.2cm} M{2.2cm} }
        \toprule
        \multirow{2}{*}{} & \multicolumn{1}{c|}{\textbf{Phase-1}} & %
        \multicolumn{3}{c}{\textbf{Phase-2}} \\
        \cmidrule{2-5}
        \textbf{Metrics}& \textbf{Rasa-based Approach} (2500)  & \textbf{Rasa-based Approach} (7500) & \textbf{Combined Approach} (2500)  & \textbf{Combined Approach} (7500) \\ \hline
        \textbf{Precision} & 0.382 & 0.388 & \textbf{0.456}  & \textbf{0.456} \\
        \textbf{Recall} & 0.26 & 0.26 & \textbf{0.31} & \textbf{0.31} \\
        \textbf{F1} & 0.309 & 0.311 & \textbf{0.369} & \textbf{0.369} \\
        \bottomrule
    \end{tabular} 
\end{table} 
%\end{center}
%\smallskip

For Research Fields and Methods, we carried out a qualitative evaluation against 10 randomly selected articles from Phase-1 holdout corpus. Tables \ref{tab:field} and \ref{tab:method} depict a comparison between the predicted fields and methods in Phase-1 and Phase-2. In general, our models returned a more granular output in the second phase, solely because of the modifications we made in the vocabularies. 


%\myblue{Evaluation against Phase-1 holdout} \\
%\bigskip
\begin{table}
\caption{Evaluation of Research Fields against Phase-1 holdout} \label{tab:field}
\begin{tabular}{C{1cm} C{4.5cm} C{3cm} C{3.5cm}} \hline
    \textbf{pub\textunderscore id} & \textbf{Keywords} & \textbf{Phase-1}  & \textbf{Phase-2} \\ \hline
     10328 & Cycling for transport, leisure and sport cyclists & Health evaluation & \textbf{Public health and health promotion} \\ \hline
     7270 & Older adult drug users, harm reduction & Health Education & \textbf{Correctional health care} \\ \hline
    6053 & Economic conditions - crime relationship, homicide & Homicide & \textbf{Gangs and crime} \\ \hline
\end{tabular}
\end{table}

\begin{table}
\caption{Evaluation of Research Methods against Phase-1 holdout} \label{tab:method}
\begin{tabular}{C{1cm} C{4.5cm} C{3cm} C{3.5cm}} \hline
    \textbf{pub\textunderscore id} & \textbf{Keywords} & \textbf{Phase-1}  & \textbf{Phase-2} \\ \hline
     10328 & Thematic content analysis & Thematic analysis & \textbf{Sidak correction} \\ \hline
     7270 & Interviews conducted face to face, finding systematic patterns or relationships among categories identified by reading the interview transcript & Qualitative interviewing & \textbf{Sampling design} \\ \hline
    6053 & Autoregressive integrated moving average (ARIMA) time-series model & Methodological pluralism & \textbf{Multivariate statistics} \\  \hline
\end{tabular}
\end{table}

\section{Discussion}
%\todo[inline]{challenges encountered}
Throughout the course of this competition, we encountered several challenges and limitations in all the three stages of the pipeline. In the preprocessing step, the appropriate extraction of text from PDFs turned out to be rather challenging. This was especially due to the varied formats of the publications, which made the extraction of specific sections---that contained all data relevant to our work---demanding. As mentioned before, if there was no explicit mention of the key-terms like \texttt{Abstract, Keywords, Introduction, Methodology/Data, Summary, Conclusion} in the text, then the content was saved as `reduced\_content' after applying all other preprocessing steps and filtering out any irrelevant data. \\

Our experiments suggest that the labeled publications we received for dataset detection were not uniform in the dataset mentions provided, which made it difficult to train an entity extraction model even with an increased number of training samples. Hence, there was only a slight improvement in performance when the Rasa-model was trained with 7500 publications instead of 2500. This was also why we combined the Rasa-based approach with the Simple Dataset Mention Search, so that at least the datasets that were present in the vocabulary do not get missed. 

Regarding the fields and methods, vocabularies played an immense role in their identification. The vocabularies that were provided by the SAGE publications contained some terms that were either polysemous or very high-level and therefore, were picked up by our model very often. Hence, for research methods, we created our own vocabulary containing all the relevant statistical methods, and for fields, we introduced a stopword-list of irrelevant terms and looked it up each time, before writing the result to the output file. The goal of stopword-list generation was to filter the terms that did not carry domain-specific information and sounded more like research methods than fields. Since the focus was on more granulated results, we tried to look for open ontologies for Social Science Fields and Methods and unfortunately, could not find any. It is worth mentioning that since our approach for Fields and Methods identification relied heavily upon vocabularies, it could not find any new methods or fields from the publications. 

Based on the final evaluation feedback, since our Phase-2 models did not perform as good as we expected, following are a few things that we could have done differently.
\begin{enumerate}
    \item For research methods, merging the given SAGE methods vocabulary with our manually curated vocabulary, could have resulted in methods that would have been both granular and statistical while still being relevant to the publications. Introducing a stopword-list just as we did for research field identification, could also have been another workaround. 
    \item For both fields and methods identification, we could have also tried pre-trained embeddings from glove\footnote{\url{https://nlp.stanford.edu/projects/glove}} and fastText\footnote{\url{https://fasttext.cc/docs/en/crawl-vectors.html}}.
    \item As our entity-extraction approach for Dataset Detection suffered from a limitation of inconsistent labels (i.e. datasets mentioned in the form of abbreviation, full-name, collection procedure, and keywords) in training data, we could have extended the Simple Dataset Mention Search to a pattern-oriented search based on handcrafted rules derived from sentence structure and other heuristics. 
\end{enumerate}

\section{Future Agenda}
The data provided to us in the competition displayed a cornucopia of inconsistencies even after human processing. We hence propose that machine-aided methods for computing correct and complete structured representation of publications are of central importance for scientific research such as an Open Research Knowledge Graph~\cite{DBLP:journals/corr/abs-1901-10816}\cite{DBLP:conf/esws/BuscaldiDMOR19}. Previous works on never-ending learning have shown how humans and extraction algorithms can work together to achieve high-precision and high-recall knowledge extraction from unstructured sources. In our future work, we hence aim to 
%While building models to extract information from such data, we gained insights into how we could benefit from having an even greater amount of publications. With the number of publications growing every year and new innovations making their way in, it is imperative to have a system that could update itself with the desired information and hence, record the metadata of ongoing research. To this end, 
populate a \textbf{scientific knowledge graph} based on never-ending learning. The methodology we plan to develop will be domain-independent and rely on active learning to classify, extract, link and publish scientific research artifacts extracted from open-access papers.  % comprising all the scientific domains and not just the field of social sciences. 
Inconsistency will be remedied by ontology-based checks learned from other publications such as SHACL constraints which can be manually or automatically added.\footnote{\url{https://www.w3.org/TR/shacl/}}
The resulting graphs will
\begin{itemize}
    %\item follow the concept of never-ending learning.
    \item rely on advanced distributed storage for RDF to scale to the large number of publications available;
    \item be self-feeding, i.e., crawl the web for potentially relevant content and make this content available for processing;
    \item be self-repairing, i.e., be able to update previous extraction results based on insights gathered from new content;
    \item be weakly supervised by humans (e.g. authors of publications), who would assist in correcting wrong hypotheses, thereby leveraging semi-supervised learning;
    \item provide standardized access via W3C Standards such as SPARQL.
\end{itemize}

Having such knowledge graphs would make it easier for the researchers (both young and veteran) to easily follow along with their domain of fast-paced research and eliminate the need to manually update the domain-specific ontologies for fields, methods and other metadata as new research innovations come up. 

%\end{multicols}

\section{Appendix}
The code and documentation for all our submissions can be found here: 
\url{https://github.com/dice-group/rich-context-competition}. 


\bibliographystyle{plain}
\bibliography{references} 
% \todo[inline]{ First, the judges will take a random set of publications and score precision and recall based on that set.   Second, as our technical judges suggested, since this phase of the competition is to be much more collaborative, you to help us design the evaluation.  As such, we ask that your team supply us with a short, one-page description of your planned approach for working on the Phase II challenge. We want you to tell us (broadly) your approach – your secret sauce – and tell us how would you like to be evaluated.  }
% \todo[inline]{@Rricha/Nikit: Combine text below with stuff from here, i.e. what we actually used: Preprocessing:
% Handling of words that get split by hyphens
% The identification and extraction of certain sections (e.g., title, abstract, methodology, etc.)
% Removal of non-text objects (e.g., graphs, equations, etc.) 

% Extraction of dataset mentions
% (Datasets) René Pattern-based approaches on the dependency trees of sentences [2] using handwritten as well as learned rules
% (Datasets) Daniel does RASA NLU based NER
% Merging of the two approach

% Research fields and methods
% Rricha: prepare a file with all NNP for all docs
% Consider preprocessing from above
% Compute parse tree for all sentences in D
% Gather all NPs $noun_1 … noun_m$
% Including sub-phrases of longer phrases
% Micha/Nikit: Vectorize dictionaries we received where every label comprises a set of words ${w_1 … w_n}$, create $v(w_1 … w_n) = \sum\limits_{i=1}^n v(w_i) $
% Where $v(w_i)$ is the vector of word$w_i$ in the embedding space
% Micha/Nikit:  Given a document D
% Compute $cos(v(noun_i), v(w_dict_j))$ for all NPs in $D$ and all $w_dict$ in the dictionary
% Select highest score $w_dict$ as field/method
% }

% We preprocess the plain text of the documents. In detail, this includes the handling of words that get split by hyphens, the identification and extraction of certain sections (e.g., title, abstract, methodology, etc.), as well as the removal of non-text objects (e.g., graphs, equations, etc.) from these sections.

% For the extraction of the research field, method and dataset mentions, we will follow two different pattern-based approaches on sentences [1] and on the dependency trees of sentences [2] using handwritten as well as learned rules. Since authors generally include the main contribution of the paper in the title, we would use titles (given in the article\_metadata) and abstracts to find the research fields and research methods (and also the dataset, if possible). In addition to this, we would also apply these approaches to the methodology and data section for the identification of research methods and dataset mentions.  
% While exploring the publications data, we observed that a dataset mention is always followed or preceded by an explicit mention of terms like ‘data’, ‘dataset’ or the likes. These terms together with the functional keywords like ‘use’, ‘utilize’, ‘gather’, etc. would act as a seed-set for our pattern matching approach for identifying datasets. If the identified dataset is present in the given dataset vocabulary, we will find all the dataset mentions using the same pattern matching approaches on the remaining sections of the paper (results, conclusions, discussions, summary, etc) for generating the publication-dataset relation file, else we will add it to the dataset mention output file. 

% After extracting the mentions, we will disambiguate them using AGDISTIS [3]. Therefore, we will create a knowledge base containing the given research fields, methods and datasets as RDF resources with connections gathered from the training data. AGDISTIS will use the trigram similarity between the mentions and the known research fields, methods and datasets as well as their connections to each other to either link the mentions to a known resource or create a new resource.

% For our experiments, we will use a Linux-based (Ubuntu 18.04.1 LTS) system with 8 Intel Xeon CPU E5-2620 v4 @ 2.1GHz cores (=16 virtual cores) and 128 GB RAM. 


% Link to Github?: https://github.com/dice-group/rich-context-competition
\end{document}
