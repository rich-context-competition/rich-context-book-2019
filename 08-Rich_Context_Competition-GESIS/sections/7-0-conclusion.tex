\subsection{Conclusion}
\label{sec:conclusion}
This chapter has provided an overview on our solutions submitted to the Rich Context Competition 2018. Aimed at improving search, discovery and interpretability of scholarly resources, we are addressing three distinct tasks all aimed at extracting structured information about research resources from scientitifc publications, namely the extraction of dataset mentions, the extraction of mentions of research methods and the classification of research fields. 

In order to address all aforementioned challenges, our pipelines make use of a range of preprocessing techniques together with state-of-the-art NLP methods as well as supervised machine learning approaches tailored towards the specific nature of scholarly publications as well as the dedicated tasks. In addition, background datasets have been used to facilitate supervision of methods at larger scale.

Our results indicate both significant opportunities for automating the aforementioned three tasks but also their challenging nature, in particular given the lack of publicly available gold standards for training and testing. Aggregating and publishing such data has been identified as important activity for future work, and is a prerequisite for significantly advancing state-of-the-art methods.

%TODO Stefan will expand this with more comprehensive discussion/future directions section at some point.
