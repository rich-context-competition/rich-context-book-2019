%% LyX 2.3.2 created this file.  For more info, see http://www.lyx.org/.
%% Do not edit unless you really know what you are doing.
\documentclass[english]{llncs}
\usepackage[T1]{fontenc}
\usepackage[latin9]{inputenc}
\usepackage{amssymb}
\usepackage{stmaryrd}
\usepackage{graphicx}
\usepackage{setspace}
\usepackage[authoryear]{natbib}
\onehalfspacing
\usepackage{babel}
\begin{document}
\title{Dataset mention extraction in scientific articles using a BiLSTM-CRF
model}
\author{Tong Zeng$^{1,2}$ and Daniel Acuna$^{1}$\thanks{Corresponding author: deacuna@syr.edu}}
\institute{$^{1}$School of Information Studies, Syracuse University, Syracuse,
USA\\
$^{2}$School of Information Management, Nanjing University, Nanjing,
China}
\maketitle
\begin{abstract}
Datasets are critical for scientific research, playing a role in replication,
reproducibility, and efficiency. Researchers have recently shown that
datasets are becoming more important for science to function properly,
even serving as artifacts of study themselves. However, citing datasets
is not a common or standard practice in spite of recent efforts by
data repositories and funding agencies. This greatly affects our ability
to track their usage and importance. A potential solution to this
problem is to automatically extract dataset mentions from scientific
articles. In this work, we propose to achieve such extraction by using
a neural network based on a BiLSTM-CRF architecture. Our method achieves
$F_{1}=0.883$ in social science articles released as part of the
Rich Context Dataset. We discuss limitations of the current datasets
and propose modifications to the model to be done in the future.
\end{abstract}


\section{Introduction}

Science is fundamentally an incremental discipline that depends on
previous scientist' work. Datasets form an integral part of this process
and therefore should be shared and cited as any other scientific output.
This ideal is far from reality; the credit that datasets currently
receive does not correspond to their actual usage. One of the issues
is that there is no standard approach for citing them. Interestingly,
while datasets are still used and mentioned in articles, we lack methods
to extract such mentions and properly reconstruct dataset citations.
The Rich Context Competition challenge aims at closing this gap by
inviting scientists to produce automated dataset mention and linkage
detection algorithms. In this article, we detail our proposal to solve
the dataset mention step. Our approach attempts to provide a first
approximation to better give credit and keep track of datasets and
their usage.

The problem of dataset extraction has been explored before. \citet{ghavimiIdentifyingImprovingDataset2016}
and \citet{ghavimiSemiautomaticApproachDetecting2017} use a relatively
simple tf-idf representation with cosine similarity for matching dataset
identification in social science articles. Their method consists of
four major steps: preparing a curated dictionary of typical mention
phrases, detecting dataset references, and ranking matching datasets
based on cosine similarity of tf-idf representations. This approach
achieved an impressive $F_{1}=0.84$ for mention detection and $F_{1}=0.83$,
for matching. \citet{singhalDataExtractMining2013} proposed a method
using normalized Google distance to screen whether a term is in a
dataset. However, this method relies on external services and is not
computational efficient. They achieve a good $F_{1}=0.85$ using Google
search and $F_{1}=0.75$ using Bing. A somewhat similar project was
proposed by \citet{luDatasetSearchEngine2012}. They built a dataset
search engine by solving the two challenges: identification of the
dataset and association to a URL. They build a dataset of 1000 documents
with their URLs, containing 8922 words or abbreviations representing
datasets. They also build a web-based interface. This shows the importance
of dataset mention extraction and how several groups have tried to
tackle the problem.

In this article, we describe a method for extracting dataset mentions
based on a deep recurrent neural network. In particular, we used a
Bidirectional Long short-term Memory (BiLSTM) sequence to sequence
model paired with a Conditional Random Field (CRF) inference mechanism.
We tested our model on a novel dataset produced for the Rich Context
Competition challenge. We achieve a relatively good performance of
$F_{1}=0.883$. We discuss the current noise and duplication present
in the dataset and limitations of our model.

\section{The dataset}

The Rich Context Dataset challenge was proposed by the New York University's
Coleridge Initiative \citep{richtextcompetition}. The challenge comprised
several phases, and participants moved through the phases depending
on their performance. We only analyze data of the first phase. This
phase contained a list of datasets and a labeled corpus of around
5K publications. Each publication was labeled indicating whether a
dataset was mentioned within it and which part of the text mentioned
it. The challenge used the accuracy for measuring the performance
of the competitors and also the quality of the code, documentation,
and efficiency.

We adopt the CoNLL 2003 format \citep{tjong2003introduction} to annotate
whether a token is a part of dataset mention. Concretely, we use we
use B-DS denotes a token is the first token of a dataset mention,
I-DS denote a token is inside of dataset mention, and O means a token
is not a part of dataset mention. We then put each token and its corresponding
labels in one line and use a empty line as separator between sentences.
All the sentences was split by 70\%, 15\%, 15\% as training set, validation
set and testing set.

\section{The Proposed Method}

\subsection{Overall view of the architecture}

In this section, we propose a model for detecting mentions based on
a BiLSTM-CRF architecture. At a high level, the model uses a sequence-to-sequence
recurrent neural network that produces the probability of whether
a token belongs to a dataset mention. The CRF layer takes those probabilities
and estimates the most likely sequence based on constrains between
label transitions (i.e., mention--to--no-mention--to-mention has
low probability). While this is a standard architecture for modeling
sequence labeling, the application to our particular dataset and problem
is new. 

We now describe in more detail the choices of word representation,
hyper-parameters, and training parameters. A schematic view of the
model is in Fig \ref{fig:NetworkArchitecture} and the components
are as follows:
\begin{enumerate}
\item Input Layer: input the sequence of tokens to the network;
\item Embedding layer: mapping each token into fixed sized vector representation
based on fasttext (200-dimensional vectors, \citet{pennington2014glove})
\item One BiLSTM layer: make use of Bidirectional LSTM network to capture
the high level representation of the whole token sequence input (200
dimensions per direction, totally 400 output units)
\item Dense layer: project the output of the previous layer to a low dimensional
vector representation of the the distribution of labels.
\item CRF layer: find the most likely sequence of labels.
\end{enumerate}
\begin{figure}
\begin{centering}
\includegraphics[width=5cm]{img/bilstm-crf-model}
\par\end{centering}
\caption{\label{fig:NetworkArchitecture}Network Architecture of BiLSTM-CRF
model}
\end{figure}


\subsection{Word Embedding}

The embedding is the first layer of our network and it is responsible
for mapping the word from string into vectors of numbers as the input
for other layers on top. For a given sentence $S$, we first convert
it into a sequence consisting of $n$ tokens, $S=\{c_{1},c_{2},\cdots,c_{n},\}$
. For each token $c_{i}$we lookup the embedding vector $x_{i}$ from
a word embedding matrix $M^{tkn}\in\mathbb{R}^{d|V|}$, where the
$d$ is the dimension of the embedding vector and the $V$ is the
Vocabulary of the tokens. In this paper, the matrix $M^{tkn}$ is
initialized by a pre-trained embedding, but will be updated by learning
from our corpus.

\subsection{LSTM}

Recurrent neural network (RNN) is a powerful tool to capture features
from sequential data, such as temporal series, and text. RNN could
capture long-distance dependency in theory but it suffers from the
gradient exploding/vanishing problems \citep{pascanu2013difficulty}.
The Long short-term memory (LSTM) architecture was proposed by \citet{hochreiter1997long}
and it is a variant of RNN which copes with the gradient problem.
LSTM introduces several gates to control the proportion of information
to forget from previous time steps and to pass to the next time step.
Formally, LSTM could be described by the following equations:

\begin{equation}
i_{t}=\sigma(W_{i}x_{t}+W_{i}h_{t-1}+b_{i})
\end{equation}
\begin{equation}
f_{t}=\sigma(W_{f}x_{t}+W_{f}h_{t-1}+b_{f})
\end{equation}
\begin{equation}
g_{t}=tanh(W_{g}x_{t}+W_{g}h_{t-1}+b_{g})
\end{equation}
\begin{equation}
o_{t}=\sigma(W_{o}x_{t}+W_{o}h_{t-1}+b_{o})
\end{equation}
\begin{equation}
c_{t}=f_{t}\bigotimes c_{t-1}+i_{t}\bigotimes g_{t}
\end{equation}
\begin{equation}
h_{t}=o_{t}\bigotimes tanh(c_{t})
\end{equation}
where the $\sigma$ is the sigmoid function, $\bigotimes$ denotes
the dot product, $b$ is the bias, $W$ is the parameters, $x_{t}$
is the input at time $t$, $c_{t}$ is the LSTM cell state at time
$t$ and $h_{t}$ is hidden state at time $t$. The $i_{t}$, $f_{t}$,
$o_{t}$ and $g_{t}$ are named as input, forget, output and cell
gates respectively, they control the informations to keep in its state
and pass to next step.

LSTM get information from the previous steps, that is left context
in our task. However, it is important to consider the information
in the right context. A solution of this information need is bidirectional
LSTM \citep{graves2013speech}. The idea of Bi-LSTM is using two LSTM
layers and feed in each layer with sequence forwards and backwards
separately, and then concatenate the hidden states of the two LSTM
to modeling both the left and right contexts

\begin{equation}
h_{t}=[\overrightarrow{h_{t}}\varoplus\overleftarrow{h_{t}}]
\end{equation}

Finally, the outcomes of the states are taken by a Conditional Random
Field (CRF) layer that takes into account the transition nature of
the beginning, intermediate, and ends of mentions. For a reference
of CRF, refer to \citep{lafferty2001conditional}

\section{Results}

In this work, we wanted to propose a model for the Rich Context Competition
challenge. We propose a relatively standard architecture based on
a BiLSTM-CRF recurrent neural network. We now describe the results
of this network on the dataset provided by the competition.

For all of our results, we use $F_{1}$ as the measure of choice.
This measure is the harmonic average of the precision and recall and
it is the standard measure used in sequence labeling tasks. This metric
varies from 0 to 1, and the unit is the highest possible value. Our
method achieved a relatively high $F_{1}$ of 0.883 for detecting
mentions, in line with previous studies.

We found significant limitations to the dataset, and we expect these
limitations to affect the linkage step (not done in this article).
While we are proposing a model for such step, we found that it would
be challenging to do so given the quality of the annotations. Specifically,
we found significant duplication of labels. The first issue is that
mentions are nested (e.g. HRS, RAND HRS, HRS DATA, RAND HRS DATA are
nested and linked to the same dataset). The second issue is that for
the same mention text, several, different datasets were linked (e.g.
the term CPS is linked to 57 datasets, the term NHANES is linked to
32 datasets). This adds noise to the linkage process. In fact, most
of the mentions have ambiguous relationships to datasets. In particular,
only 17,267 (16.99\%) mentions are linked to one dataset, 15,292 (15.04\%)
mentions are listed to two datasets, and 12,624 (12.42\%) are linked
to three datasets. We found that there were some extreme cases, where
for example there are several mentions linked to more than one hundred
datasets. If these difficulties are not overcome, then the predictions
from the linkage process will be noisy and therefore impossible to
tell apart.

\section{Conclusion}

In this work, we report a high accuracy model for the problem of detecting
dataset mentions. Because our method is based on a standard BiLSTM-CRF
architecture, we expect that updating our model with recent developments
in neural networks would only benefit our results. We also provide
some evidence of how difficult we believe the linkage step of the
challenge could be if the dataset noise are not lowered. 

One of the shortcomings of our approach is that the architecture is
lacking some modern features of RNN networks. In particular, recent
work has shown that attention mechanisms are important especially
when the task requires spatially distant information, such as this
one. These benefits could also translate to better linkage. We are
exploring new architectures using self-attention and multiple-head
attention. We hope to explore these approaches in the near future.

Our proposal, however, is surprisingly effective. Because we have
barely modified a general RNN architecture, we expect that our results
will generalize relatively well either to the second phase of the
challenge or even to other disciplines. We would emphasize, however,
that the quality of the dataset has a great deal of room for improvement.
Given how important this task is for the whole of science, we should
try to strive to improve on the quality of these datasets so that
techniques like this one can be more broadly applied. The importance
of dataset mention and linkage therefore could be fully appreciated
by the community. 

\section*{Acknowledgements}

Tong Zeng was funded by the China Scholarship Council \#201706190067.
Daniel E. Acuna was funded by the National Science Foundation awards
\#1646763 and \#1800956.

\bibliographystyle{apalike}
\bibliography{rcc-06}

\end{document}
