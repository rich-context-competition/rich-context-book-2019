%% LyX 2.3.3 created this file.  For more info, see http://www.lyx.org/.
%% Do not edit unless you really know what you are doing.
\documentclass[english]{llncs}
\usepackage[T1]{fontenc}
\usepackage[latin9]{inputenc}
\usepackage{amssymb}
\usepackage{stmaryrd}
\usepackage{graphicx}
\usepackage{setspace}
\usepackage[authoryear]{natbib}
\onehalfspacing

\makeatletter

%%%%%%%%%%%%%%%%%%%%%%%%%%%%%% LyX specific LaTeX commands.
%% Because html converters don't know tabularnewline
\providecommand{\tabularnewline}{\\}
%% A simple dot to overcome graphicx limitations
\newcommand{\lyxdot}{.}


\makeatother

\usepackage{babel}
\begin{document}
\title{Finding datasets in publications: The Syracuse University approach}
\subtitle{Dataset mention extraction in scientific articles using a BiLSTM-CRF
model}
\author{Tong Zeng$^{1,2}$ and Daniel Acuna$^{1}$\thanks{Corresponding author: deacuna@syr.edu}}
\institute{$^{1}$School of Information Studies, Syracuse University, Syracuse,
USA\\
$^{2}$School of Information Management, Nanjing University, Nanjing,
China}
\maketitle
\begin{abstract}
Datasets are critical for scientific research, playing a role in replication,
reproducibility, and efficiency. Researchers have recently shown that
datasets are becoming more important for science to function properly,
even serving as artifacts of study themselves. However, citing datasets
is not a common or standard practice in spite of recent efforts by
data repositories and funding agencies. This greatly affects our ability
to track their usage and importance. A potential solution to this
problem is to automatically extract dataset mentions from scientific
articles. In this work, we propose to achieve such extraction by using
a neural network based on a BiLSTM-CRF architecture. Our method achieves
$F_{1}=0.885$ in social science articles released as part of the
Rich Context Dataset. We discuss future improvements to the model
and applications beyond social sciences.
\end{abstract}


\section{Introduction}

Science is fundamentally an incremental discipline that depends on
previous scientist's work. Datasets form an integral part of this
process and therefore should be shared and cited as any other scientific
output. This ideal is far from reality: the credit that datasets currently
receive does not correspond to their actual usage\citep{datarank}.
One of the issues is that there is no standard for citing datasets,
and even if they are cited, they are not properly tracked by major
scientific indices. Interestingly, while datasets are still used and
mentioned in articles, we lack methods to extract such mentions and
properly reconstruct dataset citations. The Rich Context Competition
challenge aims at closing this gap by inviting scientists to produce
automated dataset mention and linkage detection algorithms. In this
article, we detail our proposal to solve the dataset mention step.
Our approach attempts to provide a first approximation to better give
credit and keep track of datasets and their usage.

The problem of dataset extraction has been explored before. \citet{ghavimiIdentifyingImprovingDataset2016}
and \citet{ghavimiSemiautomaticApproachDetecting2017} use a relatively
simple tf-idf representation with cosine similarity for matching dataset
identification in social science articles. Their method consists of
four major steps: preparing a curated dictionary of typical mention
phrases, detecting dataset references, and ranking matching datasets
based on cosine similarity of tf-idf representations. This approach
achieved a relatively high performance, with $F_{1}=0.84$ for mention
detection and $F_{1}=0.83$, for matching. \citet{singhalDataExtractMining2013}
proposed a method using normalized Google distance to screen whether
a term is in a dataset. However, this method relies on external services
and is not computational efficient. They achieve a good $F_{1}=0.85$
using Google search and $F_{1}=0.75$ using Bing. A somewhat similar
project was proposed by \citet{luDatasetSearchEngine2012}. They built
a dataset search engine by solving the two challenges: identification
of the dataset and association to a URL. They build a dataset of 1000
documents with their URLs, containing 8922 words or abbreviations
representing datasets. They also build a web-based interface. This
shows the importance of dataset mention extraction and how several
groups have tried to tackle the problem.

In this article, we describe a method for extracting dataset mentions
based on a deep recurrent neural network. In particular, we used a
Bidirectional Long short-term Memory (BiLSTM) sequence to sequence
model paired with a Conditional Random Field (CRF) inference mechanism.
The architecture is similar to\textbf{ }chapter 6, but we only focus
on the detection of dataset mentions. We tested our model on a novel
dataset produced for the Rich Context Competition challenge. We achieve
a relatively good performance of $F_{1}=0.885$. We discuss the limitations
of our model.

\section{The dataset}

The Rich Context Dataset challenge was proposed by the New York University's
Coleridge Initiative \citep{richtextcompetition}. The challenge comprised
several phases, and participants moved through the phases depending
on their performance. We only analyze data of the first phase. This
phase contained a list of datasets and a labeled corpus of around
5K publications. Each publication was labeled indicating whether a
dataset was mentioned within it and which part of the text mentioned
it. The challenge used the accuracy for measuring the performance
of the competitors and also the quality of the code, documentation,
and efficiency.

We adopt the CoNLL 2003 format \citep{tjong2003introduction} to annotate
whether a token is a part of dataset mention. Concretely, we use the
tag DS denotes a dataset mention; The B- prefix indicates that the
token is the beginning of a dataset mention, the I- prefix indicates
the token is inside of dataset mention, and O denotes a token that
is not a part of dataset mention. We put each token and its tag (separated
by horizontal tab control character) in one line, and use the end
of line (\textbackslash n) control character as separator between
sentences (see Table \ref{tab:annotation_example}). The dataset were
randomly split by 70\%, 15\%, 15\% for training set, validation set
and testing set, respectively.

\begin{table}

\centering{}\caption{\label{tab:annotation_example}Example of a sentence annotated by
IOB tagging format.}
\begin{tabular}{cc}
\hline 
Token & Annotation\tabularnewline
\hline 
This & O\tabularnewline
\multicolumn{2}{c}{...}\tabularnewline
data & O\tabularnewline
from & O\tabularnewline
the & O\tabularnewline
Monitoring & B-DS\tabularnewline
the & I-DS\tabularnewline
Future & I-DS\tabularnewline
( & O \tabularnewline
MTF & B-DS\tabularnewline
) & O\tabularnewline
\textbackslash n & \tabularnewline
\hline 
\end{tabular}
\end{table}


\section{The Proposed Method}

\subsection{Overall view of the architecture}

In this section, we propose a model for detecting mentions based on
a BiLSTM-CRF architecture. At a high level, the model uses a sequence-to-sequence
recurrent neural network that produces the probability of whether
a token belongs to a dataset mention. The CRF layer takes those probabilities
and estimates the most likely sequence based on constrains between
label transitions (e.g., mention--to--no-mention--to-mention has
low probability). While this is a standard architecture for modeling
sequence labeling, the application to our particular dataset and problem
is new.

We now describe in more detail the choices of word representation,
hyper-parameters, and training parameters. A schematic view of the
model is in Fig \ref{fig:NetworkArchitecture} and the components
are as follows:
\begin{enumerate}
\item Character encoder layer: treat a token as a sequence of characters
and encode the characters by using a bidirectional LSTM to get a vector
representation.
\item Word embedding layer: mapping each token into fixed sized vector representation
by using a pre-trained word vector.
\item BiLSTM layer: make use of Bidirectional LSTM network to capture the
high level representation of the whole token sequence input.
\item Dense layer: project the output of the previous layer to a low dimensional
vector representation of the the distribution of labels.
\item CRF layer: find the most likely sequence of labels.
\end{enumerate}
\begin{figure}
\begin{centering}
\includegraphics[width=0.8\textwidth]{img/Figure2\lyxdot 1}
\par\end{centering}
\caption{\label{fig:NetworkArchitecture}Network Architecture of BiLSTM-CRF
network}
\end{figure}


\subsection{Character encoder}

Similar to the bag of words assumption, a word could be composed of
characters sampled from a bag of characters. Previous research \citep{santos2014learning,jozefowicz2016exploring}
has shown that the use of character-level embedding could benefit
multiple NLP-related tasks. In order to use character-level information,
we break down a word into a sequence of characters, then build a vocabulary
of characters. We initialize the character embedding weights using
the vocabulary size of a pre-defined embedding dimension, then update
the weights during the training process to get the fixed-size character
embedding. Next, we feed a sequence of the character embedding into
an encoder (a bidirectional LSTM network) to produce a vector representation
of a word. By using a character encoder, we can solve the out-of-vocabulary
problem for pre-trained word embedding, as every word could be composed
of characters.

\subsection{Word Embedding}

The word embedding layer is responsible for storing and retrieving
the vector representation of words. Concretely, the word embedding
layer contains a word embedding matrix $M^{tkn}\in\mathbb{R}^{|V|d}$,
where the $V$ is the vocabulary of the tokens and the $d$ is the
size of the embedding vector. The embedding matrix was initialized
by a pre-trained GloVe vectors \citep{pennington2014glove}, and updated
by learning from the data. In order to retrieve from the embedding
matrix, we first convert a given sentence into a sequence of tokens,
then for each token we lookup the embedding matrix to get its vector
representation. Finally, we get a sequence of vectors as input for
the encoder layer.

\subsection{LSTM}

The Recurrent Neural Network (RNN) is a type of artificial neural
network which takes the output of previous step as input of the current
step recurrently. This recurrent nature allows it to learn from sequential
data, for example, the text which consists of a sequence of works.
RNN could capture contextual information in variable-length sequences
in theory but it suffers from gradient exploding/vanishing problems
\citep{pascanu2013difficulty}. The Long Short-Term Memory (LSTM)
architecture was proposed by \citet{hochreiter1997long} to cope with
these gradient problems. Similar to standard RNN, the LSTM network
also has a repeating module called LSTM cell. The cell remembers information
over arbitrary time steps because it allows information to flow along
it without change. The cell state is regulated by a forget gate and
an input gate which control the proportion of information to forget
from a previous time step and to remember for a next time step. Also,
there is a output gate controlling the information to flow out of
the cell. The LSTM could be defined formally by the following equations:

\begin{equation}
i_{t}=\sigma(W_{i}x_{t}+W_{i}h_{t-1}+b_{i})
\end{equation}
\begin{equation}
f_{t}=\sigma(W_{f}x_{t}+W_{f}h_{t-1}+b_{f})
\end{equation}
\begin{equation}
g_{t}=tanh(W_{g}x_{t}+W_{g}h_{t-1}+b_{g})
\end{equation}
\begin{equation}
o_{t}=\sigma(W_{o}x_{t}+W_{o}h_{t-1}+b_{o})
\end{equation}
\begin{equation}
c_{t}=f_{t}\bigotimes c_{t-1}+i_{t}\bigotimes g_{t}
\end{equation}
\begin{equation}
h_{t}=o_{t}\bigotimes tanh(c_{t})
\end{equation}
where $x_{t}$ is the input at time $t$, $W$ is the weights, $b$
is the bias. The $\sigma$ is the sigmoid function, $\bigotimes$
denotes the dot product, $c_{t}$ is the LSTM cell state at time $t$
and $h_{t}$ is hidden state at time $t$. The $i_{t}$, $f_{t}$,
$o_{t}$ and $g_{t}$ are named as input, forget, output and cell
gates respectively.

LSTM can learn from the previous steps, which is the left context
if we feed the sequence from left to right. However, the information
in the right context is also important for some tasks. The bidirectional
LSTM \citep{graves2013speech} satisfies this information need by
using two LSTMs. Concretely, one LSTM layer was fed by a forward sequence
and the other by a backward sequence. The final hidden states of each
LSTM were concatenated to model the left and right contexts

\begin{equation}
h_{t}=[\overrightarrow{h_{t}}\varoplus\overleftarrow{h_{t}}]
\end{equation}

Finally, the outcomes of the states are taken by a Conditional Random
Field (CRF) layer that takes into account the transition nature of
the beginning, intermediate, and ends of mentions. For a reference
of CRF, refer to \citep{lafferty2001conditional}

\section{Results}

In this work, we wanted to propose a model for the Rich Context Competition
challenge. We propose a relatively standard architecture based on
a BiLSTM-CRF recurrent neural network. We now describe the evaluation
metrics, hyper-parameter setting, and the results of this network
on the dataset provided by the competition.

For all of our results, we use $F_{1}$ as the measure of performance.
This measure is the harmonic average of the precision and recall and
it is the standard measure used in sequence labeling tasks. This metric
varies from 0 to 1, the higher the better. Our method achieved a relatively
high $F_{1}$ of 0.885 for detecting mentions.

\begin{table}
\caption{\label{tab:Model-search-space}Model search space and best assignments}

\centering{}%
\begin{tabular}{ccc}
\hline 
Hyper-parameter & Search space & Best parameter\tabularnewline
\hline 
number of epochs & 50 & 50\tabularnewline
patience & 10 & 10\tabularnewline
batch size & 64 & 64\tabularnewline
pre-trained word vector size & choice{[}50, 100, 200,300{]} & 100\tabularnewline
encoder hidden size & 300 & 300\tabularnewline
number of encoder layers & 2 & 2\tabularnewline
dropout rate & choice{[}0.0,0.5{]} & 0.5\tabularnewline
learning rate optimizer & adam & adam\tabularnewline
l2 regularizer & 0.01 & 0.01\tabularnewline
learning rate & 0.001 & 0.001\tabularnewline
\hline 
\end{tabular}
\end{table}

We train models using the training data and monitor the performance
using the validation data (we stop training if the performance does
not improve for the last 10 epochs). We are using the Adam optimizer
with learning rate of 0.001 and batch size equal to 64. The hidden
size of LSTM for character and word embedding is 80 and 300, respectively.
For the regularization methods, and to avoid over-fitting, we use
L2 regularization set to 0.01 and we also use dropout rate equal to
0.5. We trained 8 models with a combination of different GloVe vector
size (50, 100, 300 and 300) and dropout rate (0.0, 0.5). The hyper-parameter
settings are present in Table \ref{tab:Model-search-space}. 

\begin{table}
\caption{\label{tab:Performance-of-proposed}Performance of proposed network}

\centering{}%
\begin{tabular}{cccccc}
\hline 
Models & GloVe size & Dropout rate & Precision & Recall & $F_{1}$\tabularnewline
\hline 
m1 & 50 & 0.0 & 0.884 & 0.873 & 0.878\tabularnewline
m2 & 50 & 0.5 & 0.877 & 0.888 & 0.882\tabularnewline
m3 & 100 & 0.0 & 0.882 & 0.871  & 0.876\tabularnewline
m4 & 100 & 0.5 & 0.885 & 0.885 & \textbf{0.885}\tabularnewline
m5 & 200 & 0.0 & 0.882 & 0.884  & 0.883\tabularnewline
m6 & 200 & 0.5 & 0.885 & 0.880 & 0.882\tabularnewline
m7 & 300 & 0.0 & 0.868 & 0.886 & 0.877\tabularnewline
m8 & 300 & 0.5 & 0.876 & 0.878 & 0.877\tabularnewline
\hline 
\end{tabular}
\end{table}

The test performances are reported in Table \ref{tab:Performance-of-proposed}.
The best model is trained by word vector size 100 and dropout rate
0.5 with $F_{1}$ score 0.885 (Table \ref{tab:Performance-of-proposed}),
and it takes 15 hours 58 minutes for the training on an NVIDIA GTX
1080 Ti GPU in a computer with an Intel Xeon E5-1650v4 3.6 GHz CPU
with 128 GB of RAM.

We also found some limitations to the dataset. Firstly, we found that
mentions are nested (e.g. HRS, RAND HRS, RAND HRS DATA are linked
to the same dataset). The second issue most of the mentions have ambiguous
relationships to datasets. In particular, only 17,267 (16.99\%) mentions
are linked to one dataset, 15,292 (15.04\%) mentions are listed to
two datasets, and 12,624 (12.42\%) are linked to three datasets. If
these difficulties are not overcome, then the predictions from the
linkage process will be noisy and therefore impossible to tell apart.

\section{Conclusion}

In this work, we report a high accuracy model for the problem of detecting
dataset mentions. Because our method is based on a standard BiLSTM-CRF
architecture, we expect that updating our model with recent developments
in neural networks would only benefit our results. We also provide
some evidence of how difficult we believe the linkage step of the
challenge could be if the dataset noise are not lowered. 

One of the shortcomings of our approach is that the architecture is
lacking some modern features of RNN networks. In particular, recent
work has shown that attention mechanisms are important especially
when the task requires spatially distant information, such as this
one. These benefits could also translate to better linkage. We are
exploring new architectures using self-attention and multiple-head
attention. We hope to explore these approaches in the near future.

There are number of improvements that we can make in the future. A
first improvement is to use non-recurrent neural architectures such
as the Transformer which has shown to be faster and a more effective
learner compared to recurrent neural networks. Another improvement
would be to bootstrap information from other dataset sources such
as open access full-text articles from PubMed Open Access Subset.
This dataset contains dataset \emph{citations} \citep{datarank}---in
contrast to the most common types of citations to publications. The
location of this citations within the full-text could be exploited
to perform entity recognition. While this would be a somewhat different
problem than the one solved in this article, it would still be useful
for the goal of tracking dataset usage. In sum, by improving the learning
techniques and the dataset size and quality, we could significantly
increase the success of finding datasets in publications.

Our proposal, however, is surprisingly effective. Because we have
barely modified a general RNN architecture, we expect that our results
will generalize relatively well either to the second phase of the
challenge or even to other disciplines. We would emphasize, however,
that the quality of the dataset has a great deal of room for improvement.
Given how important this task is for the whole of science, we should
try to strive to improve the quality of these datasets so that techniques
like this one can be more broadly applied. The importance of dataset
mention and linkage therefore could be fully appreciated by the community. 

\section*{Acknowledgements}

Tong Zeng was funded by the China Scholarship Council \#201706190067.
Daniel E. Acuna was funded by the National Science Foundation awards
\#1646763 and \#1800956.

\bibliographystyle{apalike}
\bibliography{rcc-06}

\end{document}
